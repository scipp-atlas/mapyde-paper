\documentclass{article}

\usepackage{arxiv}

\usepackage[utf8]{inputenc} % allow utf-8 input
\usepackage[T1]{fontenc}    % use 8-bit T1 fonts
\usepackage{hyperref}       % hyperlinks
\usepackage{url}            % simple URL typesetting
\usepackage{booktabs}       % professional-quality tables
\usepackage{amsfonts}       % blackboard math symbols
\usepackage{nicefrac}       % compact symbols for 1/2, etc.
\usepackage{microtype}      % microtypography
\usepackage{lipsum}		% Can be removed after putting your text content
\usepackage{graphicx}
\usepackage{doi}

\usepackage[backend=biber,sorting=none]{biblatex}
\addbibresource{references.bib}

\title{an end-to-end pipeline workflow for particle physics}

%\date{September 9, 1985}	% Here you can change the date presented in the paper title
%\date{} 					% Or removing it

\author{ \href{https://orcid.org/0000-0001-6616-3433}{\includegraphics[scale=0.06]{orcid.pdf}\hspace{1mm}Giordon~Stark}\thanks{More information...} \\
  University of California, Santa Cruz \\
  Santa Cruz Institute for Particle Physics, \\
  Natural Sciences 2, Room \#337, 1156 High Street \\
  Santa Cruz, CA 95064 \\
	\texttt{gistark@ucsc.edu} \\
	%% examples of more authors
	\and
	\href{https://orcid.org/0000-0001-8392-0934}{\includegraphics[scale=0.06]{orcid.pdf}\hspace{1mm}Mike~Hance} \\
  University of California, Santa Cruz \\
  Santa Cruz Institute for Particle Physics, \\
  Natural Sciences 2, Room \#337, 1156 High Street \\
  Santa Cruz, CA 95064 \\
	\texttt{mhance@ucsc.edu} \\
	%% \AND
	%% Coauthor \\
	%% Affiliation \\
	%% Address \\
	%% \texttt{email} \\
	%% \and
	%% Coauthor \\
	%% Affiliation \\
	%% Address \\
	%% \texttt{email} \\
	%% \and
	%% Coauthor \\
	%% Affiliation \\
	%% Address \\
	%% \texttt{email} \\
}

% Uncomment to remove the date
%\date{}

% Uncomment to override  the `A preprint' in the header
%\renewcommand{\headeright}{Technical Report}
%\renewcommand{\undertitle}{Technical Report}
\renewcommand{\shorttitle}{\textit{mapyde} - an end-to-end pipeline workflow for particle physics}

%%% TODO
%%% Add PDF metadata to help others organize their library
%%% Once the PDF is generated, you can check the metadata with
%%% $ pdfinfo template.pdf
\hypersetup{
  pdftitle={mapyde - an end-to-end pipeline workflow for particle physics},
  pdfsubject={q-bio.NC, q-bio.QM},
  pdfauthor={Giordon~Stark, Mike~Hance},
  pdfkeywords={First keyword, Second keyword, More},
}

\begin{document}
\maketitle

\begin{abstract}
	\lipsum[1]
\end{abstract}


% keywords can be removed
\keywords{First keyword \and Second keyword \and More}


\section{Introduction}
\label{sec:introduction}
\lipsum[2]
\lipsum[3]


\section{Headings: first level}
\label{sec:headings-first-level}

\lipsum[4] See Section \ref{sec:headings-first-level}.

\subsection{Headings: second level}
\label{ssec:headings-second-level}
\lipsum[5]
\begin{equation}
	\xi_{ij}(t)=P(x_{t}=i,x_{t+1}=j|y,v,w;\theta)= {\frac {\alpha_{i}(t)a^{w_t}_{ij}\beta_{j}(t+1)b^{v_{t+1}}_{j}(y_{t+1})}{\sum_{i=1}^{N} \sum_{j=1}^{N} \alpha_{i}(t)a^{w_t}_{ij}\beta_{j}(t+1)b^{v_{t+1}}_{j}(y_{t+1})}}
\end{equation}

\subsubsection{Headings: third level}
\label{sssec:headings-third-level}
\lipsum[6]

\paragraph{Paragraph}
\lipsum[7]



\section{Examples of citations, figures, tables, references}
\label{sec:examples-of-citations-figures-tables-references}

\subsection{Citations}
\label{ssec:citations}
Here is an example usage of the main command (\verb+cite+): Some people thought a thing~\cite{hepdata,rivet} but other people thought something else~\cite{Conte:2012fm}. Many people have speculated that if we knew exactly why~\cite{atlasdataformat} thought this\dots

\subsection{Figures}
\label{ssec:figures}
\lipsum[10]
See Figure \ref{fig:fig1}. Here is how you add footnotes. \footnote{Sample of the first footnote.}
\lipsum[11]

\begin{figure}
	\centering
	\fbox{\rule[-.5cm]{4cm}{4cm} \rule[-.5cm]{4cm}{0cm}}
	\caption{Sample figure caption.}
	\label{fig:fig1}
\end{figure}

\begin{figure}
	\centering
  \includegraphics[width=0.5\textwidth]{{./figures/feynman/output/slepton_wino_bino}}
	\caption{Feynman Diagram of Slepton-Wino-Bino.}
	\label{fig:feynman:slepton_wino_bino}
\end{figure}

\subsection{Tables}
\label{ssec:tables}
See awesome Table~\ref{tab:table}.

The documentation for \verb+booktabs+ (`Publication quality tables in LaTeX') is available from:
\begin{center}
	\url{https://www.ctan.org/pkg/booktabs}
\end{center}


\begin{table}
	\caption{Sample table title}
	\centering
	\begin{tabular}{lll}
		\toprule
		\multicolumn{2}{c}{Part}                   \\
		\cmidrule(r){1-2}
		Name     & Description     & Size ($\mu$m) \\
		\midrule
		Dendrite & Input terminal  & $\sim$100     \\
		Axon     & Output terminal & $\sim$10      \\
		Soma     & Cell body       & up to $10^6$  \\
		\bottomrule
	\end{tabular}
	\label{tab:table}
\end{table}

\subsection{Lists}
\label{ssec:lists}
\begin{itemize}
	\item Lorem ipsum dolor sit amet
	\item consectetur adipiscing elit.
	\item Aliquam dignissim blandit est, in dictum tortor gravida eget. In ac rutrum magna.
\end{itemize}

\printbibliography

\end{document}
